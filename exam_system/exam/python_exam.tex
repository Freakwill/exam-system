%!TEX program = xelatex

\documentclass[12pt,a4paper]{ctexart}%
\usepackage[T1]{fontenc}%
\usepackage[utf8]{inputenc}%
\usepackage{lmodern}%
\usepackage{textcomp}%
\usepackage{lastpage}%
\usepackage{mathrsfs, amsfonts, amsmath, amssymb}%
\usepackage{enumerate}%
\usepackage{exampaper}%
\usepackage{fancyhdr}%
\usepackage{geometry}%
\usepackage{listings}%
\usepackage{ragged2e}%
%
\geometry{left=3.3cm,right=3.3cm,top=2.3cm,foot=1.5cm}%
\pagestyle{fancy}%
\chead{\footnotesize{\textbf{人民大学{-}雁栖湖人工智能学院考试命题纸}}}%
\cfoot{\footnotesize{第~\thepage~页~(共~\pageref{LastPage}~页)}}%
\renewcommand{\headrulewidth}{0pt}%
%
\begin{document}%
\begin{center}%
\Large{\textbf{北京雁栖湖数学与人工智能学院2025-2026学年第一学期《Python程序设计》课程期末考试试卷(A卷)}}%
\end{center}%
\begin{center}%
\begin{tabular}{cccccc}%
\textbf{考试形式:}&\multicolumn{2}{c}{($\square$闭卷/$\square$开卷)}&\multicolumn{3}{c}{}\\%
姓名&\autolenunderline{}&学号&\autolenunderline{}&教师姓名&\autolenunderline{}\\%
\end{tabular}%
\end{center}%
\vspace{10pt}%
\thispagestyle{plain}%
\noindent%
\textbf{一、填空题 (本题满分40分,每空 2 分, 共 20 空):}%
\begin{enumerate}[1)]%
\item%
Python脚本文件后缀是\autolenunderline{};%
\item%
说程序设计体现 Python 简洁明了的风格的形容词是\autolenunderline{};%
\item%
表示服务器已成功处理了客户端的请求,并将请求的资源正常返回给客户端的状态码是: \autolenunderline{};%
\item%
下载第三方库numpy, 库中有子模块random, 子模块中定义了函数rand; 如何直接导入rand: \autolenunderline{}%
\item%
写出为变量hello赋值为字符串\verb|Hello, Professor|的语句: \autolenunderline{};
%
\item%
如果函数没有返回语句,那么它默认返回: \autolenunderline{};%
\item%
Python 的格言是 \autolenunderline{}, 我用 Python;%
\item%
写出至少三种内建类型: \autolenunderline{}、\autolenunderline{}、\autolenunderline{};%
\item%
Python 网络爬虫中堪称绝配的两个第三方库/模块: \autolenunderline{}、\autolenunderline{};%
\item%
定义类方法的关键字是:\autolenunderline{};%
\item%
循环语句中,Python 的控制命令有: \autolenunderline{}、\autolenunderline{};%
\item%
下面代码打印出的结果是\autolenunderline{};
\begin{lstlisting}[language=python]
x = "Hello"
def f():
     x = "Nihao"
f()
print(x)

\end{lstlisting}
%
\item%
程序工作目录为'/beijing/haidian', 文件'bimsa.txt'在文件夹'/beijing/huairou'中,该文件的绝对路径和相对路径分别为:\autolenunderline{}、\autolenunderline{};%
\end{enumerate}%


%
\noindent%
\textbf{二、判断题 (本题满分20分,每题 2 分, 共 10 题):}%
\begin{enumerate}[1)]%
\item%
对 Python 来说,缩进具有语法功能;~~\mypar{}%
\item%
继承时,子类自动拥有父类的所有方法,没有例外,且可重写父类方法;
~~\mypar{}%
\item%
无论包\verb|mypackage|安装在哪里,\verb|import mypackage|都会自动搜索并导入该包;
~~\mypar{}%
\item%
lambda 符合标识符的文法,因此可作为变量名;
~~\mypar{}%
\item%
Python 有 case 语句;~~\mypar{}%
\item%
requests的get方法可以不受限制地访问任何网站并下载其上的资源;
~~\mypar{}%
\item%
Python 的 \verb|for/while| 循环语句和 \verb|if| 语句一样可以接 \verb|else|;~~\mypar{}%
\item%
用open函数打开文件得到的返回值是文件中的内容;~~\mypar{}%
\item%
不能在Python 的函数内部操作全局变量;~~\mypar{}%
\item%
函数内部不能调用函数自身,否则会陷入无限循环;~~\mypar{}%
\end{enumerate}%


%
\noindent%
\textbf{三、选择题 (本题满分40分,每题 4 分, 共 10 题):}%
\begin{enumerate}[1)]%
\item%
下面哪个起到流程控制作用的关键字不是Python的关键字;~~\mypar{}\\
(A) break~~(B) continue~~(C) return~~(D) goto%
\item%
学习编程的第一个命令是?~~\mypar{}\\
(A) 打印Hello World~~(B) 打印Hello Kitty~~(C) 打印Hello Kugou~~(D) 打印Hey dude%
\item%
下面关于字典\verb|card={"number":3, "suit":"club"}|的使用(语法上)错误的是?~~\mypar{}\\
(A) \verb|card.number=3|
~~(B) \verb|card["number"]="A"| 
~~(C) \verb|card[3]=card["number"]|
~~(D) \verb|card["suit"]=input()|
%
\item%
方法\verb|.item|常用与什么类型数据的遍历;~~\mypar{}\\
(A) 元组~~(B) 字典~~(C) 列表~~(D) 字符串%
\item%
定义函数 \verb|def foo(x, y, z:int=1, *args, **args): pass|。正确调用这个函数的命令是;
~~\mypar{}\\
(A) foo(1)~~(B) foo(1, 2, 3, 'hello')~~(C) foo(1, y=2, z='hello')~~(D) foo(y=3, 2, 3)%
\item%
文件操作完毕后,应该;~~\mypar{}\\
(A) 直接退出程序~~(B) 关机~~(C) 拔电源~~(D) 关闭文件对象%
\item%
Python发明人是在什么情况下发明Python的?~~\mypar{}\\
(A) 内心迷茫的时候~~(B) 在实验室里专研~~(C) 度假时候无聊打发时间~~(D) 泡咖啡馆奇思妙想%
\item%
设\verb|Card|是类名,card是其对象,则下述说法错误的是;~~\mypar{}\\
(A) Card也是变量~~(B) Card可像函数一样调用~~(C) card的类型是Card~~(D) Card本身无类型%
\item%
为学生名单列表students增加两位新同学"Mike"和"Lucy",可行的命令是?~~\mypar{}\\
(A) \verb|students.append("Mike, Lucy")|
~~(B) \verb|students.append("Mike"); students.append("Lucy")| 
~~(C) \verb|students.extend("Mike", "Lucy")|
~~(D) \verb|students + ["Mike", "Lucy"]|
%
\item%
安装包\verb|bs4|后,用户想直接在脚本中使用包中的类\verb|BeautifulSoup|,那么应该先做什么;
~~\mypar{}\\
(A) 无需导入直接使用BeautifulSoup~~(B) \verb|import bs4 as BeautifulSoup|~~(C) \verb|from bs4 import BeautifulSoup|~~(D) \verb|import BeautifulSoup| from bs4%
\end{enumerate}%


%
\end{document}