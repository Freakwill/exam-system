%!TEX program = xelatex

\documentclass[12pt,a4paper]{ctexart}%
\usepackage[T1]{fontenc}%
\usepackage[utf8]{inputenc}%
\usepackage{lmodern}%
\usepackage{textcomp}%
\usepackage{lastpage}%
\usepackage{mathrsfs, amsfonts, amsmath, amssymb}%
\usepackage{enumerate}%
\usepackage{exampaper}%
\usepackage{fancyhdr}%
\usepackage{geometry}%
\usepackage{listings}%
\usepackage{ragged2e}%
%
\geometry{left=3.3cm,right=3.3cm,top=2.3cm,foot=1.5cm}%
\pagestyle{fancy}%
\chead{\footnotesize{\textbf{人民大学{-}雁栖湖人工智能学院考试命题纸}}}%
\cfoot{\footnotesize{第~\thepage~页~(共~\pageref{LastPage}~页)}}%
\renewcommand{\headrulewidth}{0pt}%
%
\begin{document}%
\begin{center}%
\Large{\textbf{北京雁栖湖数学与人工智能学院2025-2026学年第一学期《Python程序设计》课程期末考试试卷(A卷)}}%
\end{center}%
\begin{center}%
\begin{tabular}{cccccc}%
\textbf{考试形式:}&\multicolumn{2}{c}{($\square$闭卷/$\square$开卷)}&\multicolumn{3}{c}{}\\%
姓名&\autolenunderline{}&学号&\autolenunderline{}&教师姓名&\autolenunderline{}\\%
\end{tabular}%
\end{center}%
\vspace{10pt}%
\thispagestyle{plain}%
\noindent%
\textbf{一、填空题 (本题满分40分,每空 2 分, 共 20 空):}%
\begin{enumerate}[1)]%
\item%
连接某个主机时,通常要先获得服务器的具体地址,包括\autolenunderline{}、\autolenunderline{}两部分;%
\item%
C 语言中的数组类似于 Python 中的$\autolenunderline{}$类型;%
\item%
函数返回语句的关键字是: \autolenunderline{};%
\item%
程序工作目录为'/beijing/haidian', 文件'bimsa.txt'在文件夹'/beijing/huairou'中,该文件的绝对路径和相对路径分别为:\autolenunderline{}、\autolenunderline{};%
\item%
为了实现“乘坐马斯克的星舰飞向太空”的梦想,用户可先用\autolenunderline{}下载具有该功能的\autolenunderline{} (如\verb|starship|), 然后在 Python 的脚本文件中使用\autolenunderline{}关键字导入;
%
\item%
Word文档属于\autolenunderline{}文件,用\autolenunderline{}方法写入(保存或另存为),用\autolenunderline{}方法读取;%
\item%
Python 网络爬虫中堪称绝配的两个第三方库/模块: \autolenunderline{}、\autolenunderline{};%
\item%
定义类方法的关键字是:\autolenunderline{};%
\item%
Python中进行幂运算的运算符是\autolenunderline{};
%
\item%
Python脚本属于\autolenunderline{}文件,用\autolenunderline{}方法写入(保存或另存为),用\autolenunderline{}方法读取;%
\item%
匿名函数的关键字是: \autolenunderline{};%
\item%
循环语句中,Python 的控制命令有: \autolenunderline{}、\autolenunderline{};%
\item%
写出为变量hello赋值为字符串\verb|Hello, Professor|的语句: \autolenunderline{};
%
\item%
下面代码打印出的结果是\autolenunderline{};
\begin{lstlisting}[language=python]
x = "Hello"
def f():
     x = "Nihao"
f()
print(x)

\end{lstlisting}
%
\end{enumerate}%


%
\noindent%
\textbf{二、判断题 (本题满分20分,每题 2 分, 共 10 题):}%
\begin{enumerate}[1)]%
\item%
相比传统计算机语言,Python 运行速度比较慢,但开发效率高;~~\mypar{}%
\item%
Python中没有函数指针的概念,函数的行为和普通变量是一样的;~~\mypar{}%
\item%
requests的get方法可以不受限制地访问任何网站并下载其上的资源;
~~\mypar{}%
\item%
继承时,子类通常会自动拥有父类方法,且可重写父类方法;
~~\mypar{}%
\item%
函数内部不能调用函数自身,否则会陷入无限循环;~~\mypar{}%
\item%
Python 没有 goto 语句;~~\mypar{}%
\item%
用with 语句执行完文件后,应该执行close方法以确保完成对硬盘中文件的操作;~~\mypar{}%
\item%
命令\verb|from mypackage import foo|在导入包中的变量foo同时,也会导入包\verb|mypackage|自身;
~~\mypar{}%
\item%
对 Python 来说,缩进具有语法功能;~~\mypar{}%
\item%
字符串的值一旦改变(比如接上其他字符串)其id一定改变~~\mypar{}%
\end{enumerate}%


%
\noindent%
\textbf{三、选择题 (本题满分40分,每题 4 分, 共 10 题):}%
\begin{enumerate}[1)]%
\item%
设类\verb|Hero|中定义了方法\verb|attack|,\verb|hero|是\verb|Hero|的对象。正确调用这个方法的命令是;~~\mypar{}\\
(A) hero.attack~~(B) hero['attack']~~(C) herro(attack)~~(D) attack.hero%
\item%
关于列表说法正确的是?~~\mypar{}\\
(A) 列表中的元素不能再是列表~~(B) 列表中的元素不能是元组~~(C) 列表中的元素类型必须相同~~(D) 在表示数据方面,列表和元组没有明显区别%
\item%
下面哪个起到流程控制作用的关键字不是Python的关键字;~~\mypar{}\\
(A) break~~(B) continue~~(C) return~~(D) goto%
\item%
方法\verb|.item|常用与什么类型数据的遍历;~~\mypar{}\\
(A) 元组~~(B) 字典~~(C) 列表~~(D) 字符串%
\item%
通过\verb|from PIL import Image|导入包之后,下面不可以做的是;
~~\mypar{}\\
(A) 使用变量\verb|PIL|~~(B) 使用变量\verb|Image|~~(C) 为变量\verb|Image|赋值~~(D) 为变量\verb|PIL|赋值%
\item%
存储个人信息时,最适合采用什么数据类型?~~\mypar{}\\
(A) 元组~~(B) 字典~~(C) 列表~~(D) 集合%
\item%
下面哪个是不合法的命令;~~\mypar{}\\
(A) class=520~~(B) \verb|class_ = 520|~~(C) \verb|print=520|~~(D) print(520)%
\item%
文件操作完毕后,应该;~~\mypar{}\\
(A) 直接退出程序~~(B) 关机~~(C) 拔电源~~(D) 关闭文件对象%
\item%
Python发明人是在什么情况下发明Python的?~~\mypar{}\\
(A) 内心迷茫的时候~~(B) 在实验室里专研~~(C) 度假时候无聊打发时间~~(D) 泡咖啡馆奇思妙想%
\item%
实现主机之间网络连接的标准方法是建立;~~\mypar{}\\
(A) Wifi~~(B) 接口对象~~(C) 套接字对象~~(D) 局域网%
\end{enumerate}%


%
\end{document}