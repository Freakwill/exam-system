%!TEX program = xelatex

\documentclass[12pt,a4paper]{ctexart}%
\usepackage[T1]{fontenc}%
\usepackage[utf8]{inputenc}%
\usepackage{lmodern}%
\usepackage{textcomp}%
\usepackage{lastpage}%
\usepackage{mathrsfs, amsfonts, amsmath, amssymb}%
\usepackage{enumerate}%
\usepackage{exampaper}%
\usepackage{fancyhdr}%
\usepackage{geometry}%
\usepackage{listings}%
\usepackage{ragged2e}%
%
\geometry{left=3.3cm,right=3.3cm,top=2.3cm,foot=1.5cm}%
\pagestyle{fancy}%
\chead{\footnotesize{\textbf{人民大学{-}雁栖湖人工智能学院考试命题纸}}}%
\cfoot{\footnotesize{第~\thepage~页~(共~\pageref{LastPage}~页)}}%
\renewcommand{\headrulewidth}{0pt}%
%
\begin{document}%
\normalsize%
\begin{center}%
\Large{\textbf{北京雁栖湖数学与人工智能学院2025-2026学年第一学期《Python程序设计》课程期末考试试卷(A卷)}}%
\end{center}%
\begin{center}%
\begin{tabular}{cccccc}%
\textbf{考试形式:}&\multicolumn{2}{c}{($\square$闭卷/$\square$开卷)}&\multicolumn{3}{c}{}\\%
姓名&\autolenunderline{}&学号&\autolenunderline{}&教师姓名&\autolenunderline{}\\%
\end{tabular}%
\end{center}%
\vspace{10pt}%
\thispagestyle{plain}%
\noindent%
\textbf{一、填空题 (本题满分40分,每空 2 分, 共 20 空):}%
\begin{enumerate}[1)]%
\item%
说程序设计体现 Python 简洁明了的风格的形容词是\autolenunderline{pythonic};%
\item%
运算符\autolenunderline{+}可以将多个字符串拼接起来;
%
\item%
C 语言中的数组类似于 Python 中的\autolenunderline{`list`}类型;%
\item%
C 语言中的结构体类似于 Python 中的\autolenunderline{`dict`}类型;%
\item%
程序工作目录为'/beijing/haidian', 文件'bimsa.txt'在文件夹'/beijing/huairou'中,该文件的绝对路径和相对路径分别为:\autolenunderline{../huairou/bimsa.txt}、\autolenunderline{/beijing/huairou/bimsa.txt};%
\item%
如果函数没有返回语句,那么它默认返回: \autolenunderline{None};%
\item%
循环语句中,Python 的控制命令有: \autolenunderline{break}、\autolenunderline{continue};%
\item%
读取文件内容前,一般用\autolenunderline{open}创建文件对象;%
\item%
文件操作结束时候要执行\autolenunderline{close}命令;%
\item%
创建文件对象的常用语法糖是\autolenunderline{with}语句;%
\item%
定义类方法的关键字是:\autolenunderline{def};%
\item%
Python 网络爬虫中堪称绝配的两个第三方库/模块: \autolenunderline{`requests`}、\autolenunderline{`bs4`};%
\item%
表示服务器已成功处理了客户端的请求,并将请求的资源正常返回给客户端的状态码是: \autolenunderline{200};%
\item%
下面两处\verb|print|命令打印出的结果分别是\autolenunderline{angela}、\autolenunderline{angelababy};
\begin{lstlisting}[language=python]
x = "angela"
def f(x):
    x += "baby"
    print(x)
    return x
f(x)
print(x)

\end{lstlisting}
%
\item%
下面两处\verb|print|命令打印出的结果分别是\autolenunderline{4}、\autolenunderline{4};
\begin{lstlisting}[language=python]
plans = ["学习", "旅游", "玩游戏"]
def add_plan():
    plans.append("刷视频")
    print(len(plans))  # 打印plans元素个数
add_plan()
print(len(plans))

\end{lstlisting}
%
\end{enumerate}%


%
\noindent%
\textbf{二、判断题 (本题满分20分,每题 2 分, 共 10 题):}%
\begin{enumerate}[1)]%
\item%
字符串的值一旦改变(比如接上其他字符串)其id一定改变~~\true%
\item%
用with 语句执行完文件后,应该执行close方法以确保完成对硬盘中文件的操作;~~\false%
\item%
定义类Animal之后, 用Animal()新建对象时,程序会自动调用方法\verb|__init__|;
~~\true%
\item%
用open函数打开文件得到的返回值是文件中的内容;~~\false%
\item%
相比传统计算机语言,Python 运行速度比较慢,但开发效率高;~~\true%
\item%
Python 有 \verb|do-while| 语句;~~\false%
\item%
命令\verb|from mypackage import foo|在导入包中的变量foo同时,也会导入包\verb|mypackage|自身;
~~\false%
\item%
函数内部不能调用函数自身,否则会陷入无限循环;~~\false%
\item%
Python的布尔型其实是特殊的整型;~~\true%
\item%
Python 中的函数不能作为另一个函数的参数或返回值;~~\false%
\end{enumerate}%


%
\noindent%
\textbf{三、选择题 (本题满分40分,每题 4 分, 共 10 题):}%
\begin{enumerate}[1)]%
\item%
关于内建变量解释正确的是?~~\mypar{B}\\
(A) 内建变量只能在函数内使用~~(B) 内建变量是 Python 预先定义好可直接使用的\\
(C) 内建变量不可重新赋值~~(D) 内建变量都以双下划线开头和结尾%
\item%
学习编程的第一个命令是?~~\mypar{A}\\
(A) 打印Hello World~~(B) 打印Hello Kitty\\
(C) 打印Hello Kugou~~(D) 打印Hey man%
\item%
Python 的吉祥物是;~~\mypar{D}\\
(A) 一只袋鼠~~(B) 一只企鹅~~(C) 一只章鱼~~(D) 一条蟒蛇%
\item%
下面哪个起到流程控制作用的关键字不是Python的关键字;~~\mypar{D}\\
(A) break~~(B) continue~~(C) return~~(D) goto%
\item%
将列表\verb|cmd=["pip", "install", "pygame"]|拆分为\verb|cmd="pip"|, \verb|args=["install, "pygame"]|两部分,最便捷的有效方法是?~~\mypar{D}\\
(A) \verb|cmd = cmd[0]; args = cmd[1:]|
~~(B) \verb|args = cmd[1:]; cmd = cmd[0]| 
\\
(C) \verb|cmd, *args = cmd|
~~(D) \verb|cmd, args = cmd|
%
\item%
下面哪个是Python新建对象的关键字;~~\mypar{D}\\
(A) init~~(B) new~~(C) cls~~(D) 没有关键字%
\item%
通过\verb|from PIL import Image|导入包之后,下面不可以做的是;
~~\mypar{A}\\
(A) 直接使用变量\verb|PIL|~~(B) 直接使用变量\verb|Image|\\
(C) 为变量\verb|Image|赋值~~(D) 为变量\verb|PIL|赋值%
\item%
下面哪个是Python导入包的关键字;~~\mypar{B}\\
(A) \verb|include|~~(B) \verb|import|~~(C) \verb|export|~~(D) \verb|use|%
\item%
关于下述程序,下面说法错误的是;
\begin{lstlisting}[language=python]
x = ["Hello"]
def f():
    x.append("RUC")
f()
print(" ".join(x))

\end{lstlisting}
~~\mypar{D}\\
(A) 最后打印结果为Hello RUC~~(B) 第一个\verb|x|为全局变量\\
(C) \verb|f()|中的\verb|x|为全局变量~~(D) \verb|f()|中的\verb|x|为局部变量%
\item%
定义函数 \verb|def foo(x, y, z:int=1, *args, **args): pass|。正确调用这个函数的命令是;
~~\mypar{B}\\
(A) \verb|foo(1)|~~(B) \verb|foo(1, 2, 3, 'hello')|\\
(C) \verb|foo(1, y=2, z='hello')|~~(D) \verb|foo(y=3, 2, 3)|%
\end{enumerate}%


%
\end{document}