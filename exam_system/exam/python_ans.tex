%!TEX program = xelatex

\documentclass[12pt,a4paper]{ctexart}%
\usepackage[T1]{fontenc}%
\usepackage[utf8]{inputenc}%
\usepackage{lmodern}%
\usepackage{textcomp}%
\usepackage{lastpage}%
\usepackage{mathrsfs, amsfonts, amsmath, amssymb}%
\usepackage{enumerate}%
\usepackage{analysis, algebra}%
\usepackage{exampaper}%
\usepackage{fancyhdr}%
\usepackage{geometry}%
\usepackage{ragged2e}%
\usepackage{multirow}%
%
\geometry{left=3.3cm,right=3.3cm,top=2.3cm,foot=1.5cm}%
\pagestyle{fancy}%
\chead{\textbf{人民大学{-}雁栖湖人工智能学院考试命题纸}}%
\cfoot{\footnotesize{第~\thepage~页~(共~\pageref{LastPage}~页)}}%
\renewcommand{\headrulewidth}{0pt}%
%
\begin{document}%
\normalsize%
\begin{center}%
\Large{\textbf{人民大学-雁栖湖人工智能学院第 2025/2026 学年\\第 1 学期试卷}}%
\end{center}%
\begin{center}%
\begin{tabular}{lclclc}%
课程&\multicolumn{2}{c}{\autolenunderline{\_\_\_\_\_\_\_\_\_\_\_\_\_\_}}&班级&\multicolumn{2}{c}{\autolenunderline{\_\_\_\_\_\_\_\_\_\_\_\_\_\_}}\\%
姓名&\autolenunderline{}&学号(末2位)&\autolenunderline{}&教师姓名&\autolenunderline{}\\%
\end{tabular}%
\end{center}%
\begin{center}%
\begin{tabular}{|c|c|c|c|c|c|}%
\hline%
\sws{题号}&\sws{一}&\sws{二}&\sws{三}&\sws{四}&\sws{总评}\\%
\hline%
\multirow{2}{*}{计分}&&&&&\\%
&&&&&\\%
\hline%
\end{tabular}%
\end{center}%
\thispagestyle{plain}%
\noindent 一、填空题 (每空 2 分, 共 20 分):%
\begin{enumerate}[1)]%
\item%
面向对象编程的最高哲学是 \autolenunderline{万事万物都是对象};%
\item%
C 语言中的数组类似于 Python 中的$\autolenunderline{list}$类型;%
\item%
Python 的格言是 \autolenunderline{人生苦短}, 我用 Python;%
\item%
文件操作结束时候要执行\autolenunderline{close}命令;%
\item%
循环语句中,Python 的控制命令有: \autolenunderline{break}、\autolenunderline{continue};%
\item%
Python 网络编程堪称绝配的两个第三方库: 、\autolenunderline{bs4};%
\end{enumerate}%


%
\noindent 二、判断题 (每空 2 分, 共 10 分):%
\begin{enumerate}[1)]%
\item%
Python 有 do-while 语句;~~\false%
\item%
def 符合标识符的文法,因此可作为变量名;
~~\true%
\item%
Python的布尔型其实是特殊的整型;~~\true%
\end{enumerate}%


%
\noindent 三、选择题 (每空 2 分, 共 10 分):%
\begin{enumerate}[1)]%
\item%
存储个人信息时,最适合采用什么数据类型?~~\mypar{B}\\
(A) 元组~~(B) 字典~~(C) 列表~~(D) 集合%
\item%
文件操作完毕后,应该;~~\mypar{D}\\
(A) 直接退出程序~~(B) 关机~~(C) 拔电源~~(D) 关闭文件对象%
\item%
下面哪个起到流程控制作用的关键词不是Python的关键词;~~\mypar{D}\\
(A) break~~(B) continue~~(C) return~~(D) goto%
\item%
Python 的吉祥物是;~~\mypar{D}\\
(A) 一只袋鼠~~(B) 一只企鹅~~(C) 一只章鱼~~(D) 一条蟒蛇%
\item%
学习编程的第一个命令是?~~\mypar{A}\\
(A) Hello World~~(B) Hello Kitty~~(C) Hello Kugou~~(D) 你好世界%
\end{enumerate}%


%
\noindent 四、计算题 (每题 10 分, 共 60 分):%
%
\end{document}